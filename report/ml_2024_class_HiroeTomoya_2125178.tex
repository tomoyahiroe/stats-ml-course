%#!lualatex
%#BIBTEX upbibtex jecon-example

%% Filename:            paper.tex
%% Author:              Tomoya Hiroe
%% e-mail               <hiroe-tomoya-yp@ynu.jp>
%% First-written:       <20**/**/**>
%% Description: 
%%
%############################## Main #################################

%% jlreq クラスを利用
\documentclass[article]{jlreq}


\usepackage{fancyhdr}
%% ヘッダー、フッターの設定
\pagestyle{fancy}
    \lhead{} %ヘッダ左
    \chead{} %ヘッダ中央
    \rhead{廣江友哉 2125178} %ヘッダ右.コンパイルした日付を表示
    \lfoot{} %フッタ左
    \cfoot{\thepage} %フッタ中央.ページ番号を表示
    \rfoot{} %フッタ右

% 以下は lualatex 用の設定.
\usepackage[no-math]{fontspec}
\usepackage{luatexja-preset}
\usepackage{luatexja-fontspec}

%% LaTeXやBibTeX関連のロゴ
\usepackage{bxtexlogo}
%% natbib.sty を使う.
\usepackage{natbib}
%% Fancybox
\usepackage{fancybox}
%% For screen command.
\usepackage{ascmac}
%% Font を変更 → Times系に.
\usepackage{newtxtext,newtxmath}

%% 色を付ける.
\usepackage{graphicx}
\usepackage{float}
\usepackage{color}
\definecolor{MyBrown}{rgb}{0.3,0,0}
\definecolor{MyBlue}{rgb}{0,0,0.3}
\definecolor{MyRed}{rgb}{0.6,0,0.1}
\definecolor{MyGreen}{rgb}{0,0.4,0}

%% hyperref
\usepackage[%
bookmarks=true,%
bookmarksnumbered=true,%
colorlinks=true,%
linkcolor=MyBlue,%
citecolor=MyRed,%
filecolor=MyBlue,%
urlcolor=MyGreen%
]{hyperref}

\makeatletter
\ifx\undefined\bysame
\newcommand{\BibTeX}{\textrm{B\kern-.05em\textsc{i\kern-.025em b}\kern-.08em%
T\kern-.1667em\lower.7ex\hbox{E}\kern-.125emX}}
\fi
\ifx\undefined\bysame
\newcommand{\bysameline}{\hskip.3em \leavevmode\rule[.5ex]{3em}{.3pt}\hskip0.5em}
\fi
\makeatother

\setlength{\leftmarginii}{1cm}  
\jlreqsetup{
  itemization_label_length={1\zw, i=1\zw},
  itemization_labelsep={1\zw},  
}

\setlength{\leftmargini}{2\zw}% 2+0.5
\setlength{\leftmarginii}{2\zw}% 1+0.5

\newcommand{\midashi}[1]{\noindent \textbf{#1}}
\newcommand{\midashii}[1]{\noindent \textbf{[#1]}}
\newcommand{\midashiii}[1]{\noindent \textbf{□ #1}}


%%% title, author, acknowledgement, and date
\title{統計・機械学習モデル宿題1}

\author{廣江友哉\;2125178\thanks{Email: \texttt{\href{hiroe-tomoya-yp@ynu.jp}{hiroe-tomoya-yp@ynu.jp}}}}

% \date{2017年11月13日}
\date{\today}

%#####################################################################
%######################### Document Starts ###########################
%#####################################################################
\begin{document}

%%% タイトル付ける.
\maketitle

%%% 目次を出力
\tableofcontents

%########################## Text Starts ##############################

\section{課題1}

\begin{figure}[H]
  \centering
  \includegraphics[width=100mm]{../plots/hw1/boxplot.png}
  \caption{箱ひげ図}
\end{figure}

箱ひげ図からメーカーにより価格帯の広がりに違いがあることがわかる.例えば,最低価格と最高価格のトラックのメーカーは共にGMC社である.また,製品価格の中央値が最も小さいのもGMC社である.ここからGMC社は製品価格の分散が他社と比較して大きいことがわかる.一方で,価格の広がりが最も小さいのはDodge社である.四分位範囲も狭く,最高価格は他社と比較して安く,最低価格も他社と比較して高いことがわかる.Ford社は価格の中央値が3社の中で最も高い.

\begin{figure}[H]
  \centering
  \includegraphics[width=100mm]{../plots/hw1/histogram.png}
  \caption{ヒストグラム}
\end{figure}

図2のヒストグラムは,価格を5つの階級に分けて,0ドル以上25000ドル未満の範囲でプロットしている.0ドル以上5000ドル未満の階級の度数が最も高く,19台である.元のデータは46行からなるCSVデータであるから,中央値が5000ドル以上10000ドル未満の階級に属していることがわかる.

\begin{figure}[H]
  \centering
  \includegraphics[width=100mm]{../plots/hw1/scatterplot.png}
  \caption{散布図}
\end{figure}

図3の散布図を確認すると,年々トラックの製品価格が高くなる傾向が3社ともに見て取れる.また,図1の箱ひげ図でも確認したように,GMC社の製品には3社の中で最低価格のものと最高価格のものが存在する.さらに,年代に注目すると,最も古いトラックはGMC社のもので,Ford社とDodge社が同時期からデータとして存在することがわかる.

\section{課題2}

\begin{figure}[H]
  \centering
  \includegraphics[width=100mm]{../plots/hw1/density.png}
  \caption{ヒストグラム}
\end{figure}

図4のヒストグラムは,

\bibliographystyle{./jecon}
% \bibliographystyle{./jecon}
% \bibliographystyle{jecon-new}

%% BibTeX データベースファイルの指定.
%
\addcontentsline{toc}{section}{参考文献}
% \bibliography{./jecon-example}

\end{document}
%#####################################################################
%######################### Document Ends #############################
%#####################################################################
% --------------------
% Local Variables:
% fill-column: 80
% End:
